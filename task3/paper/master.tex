\documentclass[10pt,oneside,a4paper,final,english]{memoir}

\usepackage{lscape}
\usepackage{multicol}
%\usepackage{epic,eepic}
\usepackage{latexsym}
\usepackage{verbatim}
\usepackage{listings}
\usepackage{ulem}
\usepackage{hyperref}

\let\footruleskip\undefined
\usepackage{fancyhdr}
\usepackage[final]{fixme}

\let\fref\undefined
\usepackage[plain]{fancyref}

%% FOR LOOP
\usepackage{ifthen,calc}
\newcounter{myforloopcounter}
\newcommand{\forloop}[5][1]% 
{\setcounter{#2}{#3}% 
\ifthenelse{#4}% 
{#5%
  \addtocounter{#2}{#1}% 
  \forloop[#1]{#2}{\value{#2}}{#4}{#5}% 
}%
% Else 
{}%
}% 


%% USAGE
%\forloop[step]{counter}{initial_value}{conditional}{code_block} 
\usepackage[english]{babel}
\usepackage[utf8]{inputenc}

%\selectlanguage{danish}

\lstset{language=Python,basicstyle=\small,
  columns=fullflexible}


\usepackage[pdftex]{graphicx}

\DeclareGraphicsExtensions{.jpg .png .pdf}


\usepackage{amsmath}
\usepackage{latexsym}
\usepackage{amssymb}


\usepackage[osf,sc]{mathpazo}
\usepackage{microtype}
%\usepackage{fourier}
\linespread{1.05}

%\usepackage[charter]{mathdesign}
%\usepackage{lmodern}

%\usepackage{algorithmic}
%\usepackage{algorithm}

\usepackage{amsthm}


\theoremstyle{plain}  \newtheorem{definition}{Definition}
\theoremstyle{remark} \newtheorem{lemma}{Lemma}
\theoremstyle{plain}  \newtheorem{theorem}{Theorem}
\theoremstyle{remark}  \newtheorem{example}{Example}


\newcommand{\p}{\ensuremath{^\prime}}
\DeclareGraphicsExtensions{.jpg, .eps, .png}
%%% Local Variables:
%%% mode: plain-tex
%%% TeX-master: "../master"
%%% End:

%\usepackage{algorithmic}
\usepackage{algorithm}
\usepackage[sectionbib,square]{natbib}
%\bibpunct{(}{)}{,}{a}{}{}
\setcitestyle{alpha}
%\setcitestyle{numbers,aysep={},yysep={;}}

\usepackage{datetime}

\chapterstyle{thatcher}


\usepackage[autosize,outputdir=graphs/]{dot2texi}
\usepackage{tikz}
\usetikzlibrary{shapes,arrows}


%\pagestyle{fancy}
\begin{document}
  \fontencoding{T1}
%  \fontseries{m}
%  \fontshape{n}
%  \fontsize{12}{15}
%  \selectfont

\newcommand{\ct}{\texttt}
%%%%%%%%%%%%%%%%%%%%%%%%%%%%%%%%%%%%%%%%%%%%%%%%%%%%%%%%
%                    Forside
%%%%%%%%%%%%%%%%%%%%%%%%%%%%%%%%%%%%%%%%%%%%%%%%%%%%%%%%
\makeatletter % open mode for reading @ signed variables
\def\maketitle{%
 \null
 \thispagestyle{empty}%
 \vfill
 \begin{center}\leavevmode
   \normalfont
   \LARGE{\raggedleft \@title\par}%
   \hrulefill\par
   \large{\raggedleft \subtitle\par}%
   \vskip 2cm
   {\today\par}%
 \end{center}%
 \vfill
 \begin{flushleft}
   {\large \@author } \\
   {\footnotesize \suplementInfo }
 \end{flushleft}
 \clearpage % Terminates the page here. Everything else vil be placed
            % on next page.
}
\makeatother % closing mode for reading @ signed variables
%%%%%%%%%%%%%%%%%%%%%%%%%%%%%%%%%%%%%%%%%%%%%%%%%%%%%%%%
%               Data til forside
%%%%%%%%%%%%%%%%%%%%%%%%%%%%%%%%%%%%%%%%%%%%%%%%%%%%%%%%
\title{Selection Problem $\cdot$ Week III}

\def\subtitle{Data Structures: Theory and Practice}

\author{Johan Sejr Brinch Nielsen} \def\suplementInfo{

\kern 5pt \hrule width 11pc \kern 5pt

\begin{tabular}{ll}
Email: & zerrez@diku.dk  \\
Cpr.:  & 260886-2547
\end{tabular}

% putter 5pt spacing oven over og neden under stregen
\kern 5pt \hrule width 11pc \kern 5pt

Dept. of Computer Science,  \\
University of Copenhagen

}


\maketitle
\newpage


\section*{1) Exercise 19.4-1}
For any natural number $n$ it is possible to generate a fibonacci heap
of size $n$, for which its internal tree structure is linear.

\begin{itemize}
\item Base case $n=0$: Insert an element. The heap is now of size $1$
  and linear.
\item Induction step $n \to n+1$: Suppose we have a linear fibonacci
  heap of size $n$ with $x$ as minimum. Insert three nodes $a$,
  $b$ and $c$ where $a < b < c$ and all three are lower than
  the minimum of the heap ($a$ is the new minimum). Call extract-min
  to extract $a$. The \ct{Consolidate} operation will now make a tre
  with $b$ as root and $x$ and $c$ as children. Delete $c$. We
  now have a linear fibonacci heap of size $n+1$.
\end{itemize}

\begin{tabular}{c|cc|cc}
  \ct{Insert(a,b,c)} &  & \ct{Extract-min()} &  &
  \ct{Delete(c)}\\
\begin{dot2tex}
digraph G {
  x -> y -> z;
  x a b c;
  { rank=same; x a b }
  { rank=same; y z c }
}
\end{dot2tex}
&  &
\begin{dot2tex}
digraph G {
  b -> x -> y -> z;
  b -> c;
  { rank=same; x c }
  { rank=same; y z}
}
\end{dot2tex}
&  &
\begin{dot2tex}
digraph G {
  b -> x -> y -> z;
  { rank=same; x b }
}
\end{dot2tex}
\end{tabular}


\section*{2) Problem 19-4}

A mergeable heap backed by a 2-3-4-tree. The operations could be
implemented by:

\begin{itemize}
\item \ct{Minimum}: Maintain a pointer to the smallest node to
  achieve $O(1)$.

\item \ct{Decrease-key}: Decrease the key of the leaf and update all
  the affected \ct{small} values of its parent etc. Update the
  minimum-pointer if needed. This operation is $O(\log n)$.

\item \ct{Insert}: First, find the leaf where the node is to be
  inserted by choosing the appropriate child (largest \ct{small} value
  lower than the value inserted). Now, create a new leaf for the value
  and add it to the parent of the found leaf. If the parent is full do
  splitting as in 2-3-4-trees. Update the small values and the
  minimum-pointer if the inserted value is a new minimum. Time:
  $O(\log n)$

\item \ct{Delete}: This is done as in an 2-3-4-tree. First delete the
  node. If its parent becomes empty perform the appropriate
  merging. Update the appropriate \ct{small} values. If the deleted
  key was the minimum, scan the tree for the new minimum. This
  operation is $O(\log n)$ time.

\item \ct{Extract-min}: Call \ct{Delete} on the minimum (given by the
  pointer). This is $O(\log n)$.

\item \ct{Union}: How to make a new heap $H$ from two existing heaps
  $H_0$ and $H_1$. If the two $H_0$ and $H_1$ have equal heights, make
  a new root that these as children. Set its \ct{small} value to the
  minimum of the two heaps. This case takes $O(1)$.

  If $H_0$ is tallest, find some node in $H_0$ whose subtrees are the
  same height as $H_1$ and add $H_1$ as a child.

  Finding $H_1$'s new parent in $H_0$ is $O(\log(\max(n,m))) = O(\log
  (n+m))$.
\end{itemize}




\subsection{Stubbed C++ code}
\subsubsection{File: heap.h++}
\lstinputlisting{../code/heap.h++}
\subsubsection{File: heap.c++}
\lstinputlisting{../code/heap.c++}


\section*{3) Binomial Heaps}
I have investigated some running times in binomial heaps:
\begin{itemize}
\item $n$ inserts: $O(n)$
\item $m$ deletes: $O(m \log m)$
\item $n$ inserts then $m$ deletes: $O(n + m \log m)$
\end{itemize}

I convinced that the running time of a mixed sequence of inserts and
deletes are worst case $O(N\log N)$ where $N$ is the total number of
operations.


\section*{4) Pennants }

A pennant is a heap ordered binary tree with a size of $2^r$ where $r$
is the rank. The root node has no left child and its right subtree is
a perfect binary tree. The minimum value of the tree is stored in the
root node. Consider now an operation that merges to trees of same
rank:

\ct{Merge(s, t)}\\

\noindent
\begin{figure}[h]
\centering
\begin{dot2tex}[options=-tmath]
graph G {

  2 -- 3 [label="s"];
  3 -- 4;
  3 -- 15;

  { rank=same; 2 3 }

  5 -- 6 [label="t"];
  6 -- 7;
  6 -- 42;

  { rank=same; 5 6  }
  { rank=same; 42 7 }
}
\end{dot2tex}
\caption{Example of two pennants $s$ and $t$}
\end{figure}

\begin{figure}[h]
\centering
\begin{dot2tex}[options=-tmath]
graph G {
  2 -- 3 [label="p"];
  3 -- 6;
  6 -- 7;
  6 -- 42;

  3 -- 4;
  4 -- 5;
  4 -- 15;

  { rank=same; 2 3 }
}
\end{dot2tex}
\caption{The two pennants $s$ and $t$ merged}
\end{figure}

\noindent
Two pennants can be merged by:
\begin{enumerate}
\item Initialize a new pennant $p$.
\item If $s$ is the tree with the smallest root and $t$ the other
  tree then extract the root of $s$ and let it be the new root of $p$.
\item If now Root($t$) $<$ Root($s$) then set
  Right(Root($p$)) = $t$ and Left($t$) = $s$. $p$ is now the merged
  tree. This case requires $O(1)$ time.

  If instead Root($t$) $>$ Root($s$) then swap the two and
  set Right(Root($p$)) = $t$. Set Left($t$) = $s$. The problem now is
  that Left($t$) may now be a heap-ordered tree. It's root could be
  larger than its children. We therefore need to push the down until
  it reaches a smaller node or the end. This case requires $O(r)$
  time (This is the case illustrated in the example).
\end{enumerate}

The worst case running time of the \ct{Merge} operation is thereby
$O(r)$ where $r$ is the rank of the two trees (each of size $2^r$).

The worst case situation when performing a \ct{Union} operation on two
pennant queues of size $S$ is when both queues has a pennant for each
rank $r$. In this case $S$ \ct{Merge} operations are performed. The
worst case running time of this operation is thereby:
\[ O\Big(\sum_{r=0}^S r \Big) = O(\frac{S^2+S}{2}) = O(S^2) =
O(\log^2(N)) \]

Where $N$ is the number of elements in the united queues.


\section*{5) Fibonacci Heaps}
A normal fibonacci heap has the following worst case running times for
the priority queue operations:
\begin{itemize}
\item \ct{Insert}: $O(1)$
\item \ct{Find-min}: $O(1)$
\item \ct{Extract-min}: $O(n)$
\end{itemize}

The worst case of \ct{Extract-min} is bounded by the maximum of
children the minimum-node has. If we give an upper bound of $O(\log
n)$ on this number the operation would run in worst case $O(\log
n)$. My idea is to move some of the work from the \ct{Extract-min}
onto the \ct{Insert} operation which currently runs in $O(1)$. It goes
as follows:
\begin{itemize}
\item Set an upper bound of $2\log n$ on the children any node can
  have.
\item Keep track of the count of a node's children when inserting and
  extracting using a counter.
\item If a node has $2\log n$ children when inserting a new child, cut
  the $\log n$ of them and make them roots. Run consolidate.
\end{itemize}

If one could actually maintain an invariant that guaranteed that no
node has more than $O(\log n)$ children, then \ct{Extract-min} would
have $O(\log n)$ worst case.


\end{document}


%%% Local Variables:
%%% mode: latex
%%% TeX-master: t
%%% End:
