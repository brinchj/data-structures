\documentclass[10pt,oneside,a4paper,final,english]{memoir}

\usepackage{lscape}
\usepackage{multicol}
%\usepackage{epic,eepic}
\usepackage{latexsym}
\usepackage{verbatim}
\usepackage{listings}
\usepackage{ulem}
\usepackage{hyperref}

\let\footruleskip\undefined
\usepackage{fancyhdr}
\usepackage[final]{fixme}

\let\fref\undefined
\usepackage[plain]{fancyref}

%% FOR LOOP
\usepackage{ifthen,calc}
\newcounter{myforloopcounter}
\newcommand{\forloop}[5][1]% 
{\setcounter{#2}{#3}% 
\ifthenelse{#4}% 
{#5%
  \addtocounter{#2}{#1}% 
  \forloop[#1]{#2}{\value{#2}}{#4}{#5}% 
}%
% Else 
{}%
}% 


%% USAGE
%\forloop[step]{counter}{initial_value}{conditional}{code_block} 
\usepackage[english]{babel}
\usepackage[utf8]{inputenc}

%\selectlanguage{danish}

\lstset{language=Python,basicstyle=\small,
  columns=fullflexible}


\usepackage[pdftex]{graphicx}

\DeclareGraphicsExtensions{.jpg .png .pdf}


\usepackage{amsmath}
\usepackage{latexsym}
\usepackage{amssymb}


\usepackage[osf,sc]{mathpazo}
\usepackage{microtype}
%\usepackage{fourier}
\linespread{1.05}

%\usepackage[charter]{mathdesign}
%\usepackage{lmodern}

%\usepackage{algorithmic}
%\usepackage{algorithm}

\usepackage{amsthm}


\theoremstyle{plain}  \newtheorem{definition}{Definition}
\theoremstyle{remark} \newtheorem{lemma}{Lemma}
\theoremstyle{plain}  \newtheorem{theorem}{Theorem}
\theoremstyle{remark}  \newtheorem{example}{Example}


\newcommand{\p}{\ensuremath{^\prime}}
\DeclareGraphicsExtensions{.jpg, .eps, .png}
%%% Local Variables:
%%% mode: plain-tex
%%% TeX-master: "../master"
%%% End:

%\usepackage{algorithmic}
\usepackage{algorithm}
\usepackage[sectionbib,square]{natbib}
%\bibpunct{(}{)}{,}{a}{}{}
\setcitestyle{alpha}
%\setcitestyle{numbers,aysep={},yysep={;}}

\usepackage{datetime}

\chapterstyle{thatcher}

%\pagestyle{fancy}
\begin{document}
  \fontencoding{T1}
%  \fontseries{m}
%  \fontshape{n}
%  \fontsize{12}{15}
%  \selectfont


%%%%%%%%%%%%%%%%%%%%%%%%%%%%%%%%%%%%%%%%%%%%%%%%%%%%%%%%
%                    Forside
%%%%%%%%%%%%%%%%%%%%%%%%%%%%%%%%%%%%%%%%%%%%%%%%%%%%%%%%
\makeatletter % open mode for reading @ signed variables
\def\maketitle{%
 \null
 \thispagestyle{empty}%
 \vfill
 \begin{center}\leavevmode
   \normalfont
   \LARGE{\raggedleft \@title\par}%
   \hrulefill\par
   \large{\raggedleft \subtitle\par}%
   \vskip 2cm
   {\today\par}%
 \end{center}%
 \vfill
 \begin{flushleft}
   {\large \@author } \\
   {\footnotesize \suplementInfo }
 \end{flushleft}
 \clearpage % Terminates the page here. Everything else vil be placed
            % on next page.
}
\makeatother % closing mode for reading @ signed variables
%%%%%%%%%%%%%%%%%%%%%%%%%%%%%%%%%%%%%%%%%%%%%%%%%%%%%%%%
%               Data til forside
%%%%%%%%%%%%%%%%%%%%%%%%%%%%%%%%%%%%%%%%%%%%%%%%%%%%%%%%
\title{Selection Problem $\cdot$ Week III}

\def\subtitle{Data Structures: Theory and Practice}

\author{Johan Sejr Brinch Nielsen} \def\suplementInfo{

\kern 5pt \hrule width 11pc \kern 5pt

\begin{tabular}{ll}
Email: & zerrez@diku.dk  \\
Cpr.:  & 260886-2547
\end{tabular}

% putter 5pt spacing oven over og neden under stregen
\kern 5pt \hrule width 11pc \kern 5pt

Dept. of Computer Science,  \\
University of Copenhagen

}


\maketitle
\newpage

\section*{1. Exercise 9.3-1}

The running time of the randomized selection algorithm is described by
the recurrence:
\[ n < 140:    T(n) \leq O(1) \]
\[ n \geq 140: T(n) \leq T(\lceil \frac{n}{5} \rceil) +
   T(\frac{7n}{10} + 6) + O(n) \]

The running time of this algorithm is linear, since:
\[ \frac{1n}{5} + \frac{7n}{10} + 6 = \frac{9n}{10} + 6 \]

The combined sizes of the sub-problems are $\frac{9}{10}n + 6$, which
is a fraction lower than $n$, for $n > 60$. For $n = 140$ this
combined size is $132$.

\subsection{Groups of 7}
When using groups of 7 the above equation for the running time with $n
\geq 140$ becomes:

\[ 5 \Big( \lceil
   \frac{1}{2} \lceil \frac{n}{7} \rceil \rceil - 2 \Big) \geq
   \frac{5n}{14} - 10\]

Select is then called recursively on at most:
\[ n - (\frac{5n}{14} - 10) = \frac{9n}{14} + 10 \]

Which yields the running time:
\[ T(n) = T(\frac{1n}{7}) + T(\frac{9n}{14} + 10) + O(n) \]

This choice of group size yields linear running time, since:
\[ \frac{1n}{7} + \frac{9n}{14} + 10 = \frac{11}{14}n + 10\]

Again, the sub-problem is only a fraction of the original problem for
$n \geq 47$.

\subsection{Groups of 3}
When using groups of 3 the above equation for the running time with $n
\geq 140$ becomes:

\[ 1 \Big( \lceil
   \frac{1}{2} \lceil \frac{n}{3} \rceil \rceil - 2 \Big) \geq
   \frac{n}{6} - 2\]

Select is then called recursively on at most:
\[ n - (\frac{n}{6} - 2) = \frac{5n}{6} + 2 \]

Which yields the running time:
\[ T(n) = T(\frac{1n}{3}) + T(\frac{5n}{6} + 2) + O(n) \]

This choice of group size yields linear running time, since:
\[ \frac{1n}{3} + \frac{5n}{6} + 2 = \frac{7}{6}n + 2\]

Here the two sub-problems yields an even larger problem than the
original. The recursion is no longer running in linear time.



\section*{2. Problem 9-4: Analysis of Randomized Selection}
\subsection{a) Number of comparisons}

Let $E[X_{ijk}]$ be a measure for the number of expected comparisons
between the $i$th and $j$th largest element, when locating the $k$th
largest element.

I will now find an exact expression for $E[X_{ijk}$. First, observe
that
\begin{itemize}
\item $k < i$ gives $i - k + 1$ elements (from $k$ to $i$) that if
  chosen as pivot will cut $z_i$ and $z_j$ out of the problem (they
  are in the discarded subproblem). In this case, $z_i$ and $z_j$ will
  be compared if one of them are chosen as first pivot among $j - k +
  1$ elements (from $k$ to $j$).
\item $i \leq k \leq j$ gives $j-i-1$ elements that will cut either
  $z_i$ or $z_j$ out of the problem. In this case, $z_i$ and $z_j$
  will be compared if one of them are chosen as pivot among $j-i+1$
  elements.
\item $j < k$ is symmetric to the first case. $z_i$ and $z_j$ is
  compared if one of them are chosen as pivot among $k-i+1$ elements.
\end{itemize}

Combined, these three cases yields the following number of expected
comparisons:
\[ E[X_{ijk}] = \frac2{\max(j-i+1, j-k+1, k-i+1)} \]

\subsection{Total Number of Comparisons}
\[ E[X_{k}] = \sum_{i=1}^{n-1} \sum_{j=i+1}^n \frac2{\max(j-i+1, j-k+1,
  k-i+1)} = \]
\[
\sum_{i=1}^k \sum_{j=k}^n \frac2{j-i+1} +
\sum_{i=1}^{k-2} \sum_{j=i+1}^{k-1} \frac2{k-i+1} +
\sum_{i=k+1}^{n-1} \sum_{j=i+1}^n \frac2{j-k+1} = \]
\[ 2 \Big(\sum_{i=1}^k \sum_{j=k}^n \frac1{j-i+1} +
\sum_{i=1}^{k-2} \frac{k-i-1}{k-i+1} +
\sum_{j=k+1}^{n} \frac{j-k-1}{j-k+1}
 \Big)\]

Clearly,
\[ E[X_k] \leq 2 \Big(\sum_{i=1}^k \sum_{j=k}^n \frac1{j-i+1} +
\sum_{i=1}^{k-2} \frac{k-i-1}{k-i+1} +
\sum_{j=k+1}^{n} \frac{j-k-1}{j-k+1}
 \Big)\]

\subsection{Linear Running Time}
\[ E[X_k] =
2 \Big(\sum_{i=1}^k \sum_{j=k}^n \frac1{j-i+1} +
\sum_{i=1}^{k-2} \frac{k-i-1}{k-i+1} +
\sum_{j=k+1}^{n} \frac{j-k-1}{j-k+1}
 \Big) \leq \]
\[
2 \Big(\sum_{i=1}^k \sum_{j=k}^n \frac1{j-i+1} +
\sum_{i=1}^{k-2} 1 +
\sum_{j=k+1}^{n} 1
 \Big) \leq \]
\[
2 \Big(\sum_{i=1}^k \sum_{j=k}^n \frac1{j-i+1} +
(k-2) +
(n - k+1)
 \Big) \leq \]


\[ 2 \Big(\sum_{i=1}^{\min(k,n-k)} \frac1{k} +
\sum_{i=2}^{k} \sum_{j=k}^{i+k-1} \frac1{j-i+1} +
\sum_{i=1}^{k-1} \sum_{i=k+i+1}^n \frac1{j-i+1} +
n - 1
 \Big) \leq \]


\[ 2 \Big(
\frac{n}{2k} +
\sum_{i=1}^{k-1} \sum_{i=k+i+1}^n \frac1{j-i+1} +
\sum_{i=2}^{k} \sum_{i=k}^{i+k-1} \frac1{j-i+1} +
n - 1
 \Big) \]

Well, that's all i got for now :-)


\section*{3. Selection Algorithm on Multi-Sets}
One way to maintain the $O(n)$ running time on multi-sets (sets with
non-distinct elements) is:
\begin{enumerate}
\item Find the median of the medians, as in the original algorithm
\item Partition the elements in two using the found median as pivot
\item If the left section (lower-than) is chosen, proceed as usual (no
  elements in this set is equal to the pivot)
\item If the right section (greater-than or equal-to) is chosen: 1)
  move all equal-to element to the beginning of the section with a
  linear scan and 2) if the needed median is available return it,
  otherwise recurse on the rest of the right section
\end{enumerate}

Given the $<$ (less-than) operator one can test if an element $a$ is
equal to $b$ by $!(a < b \vee b < a)$. When moving equal elements in
the right section case, the equality of element $e$ can be
accomplished by $!(p < e)$, since $e \geq p$ is already known and
\[e\geq p \wedge !(p<e) \Rightarrow e \geq p \wedge p \leq e \Rightarrow p = e\]

This \textit{trick} of locating equal elements ensures that recursion
only occurs on either strictly lower or strictly greater elements
(never equal). This maintains the upper bounds on the size of the
remaining elements maintaining $O(n)$ running time.

\section*{4. Deterministic Linear-Time Selection Algorithm}
I will now investigate the performance of deterministic linear-time
selection algorithm. I benchmark this algorithm along with randomize
selction (RND), STL selection, STL sort and selection using MergeSort.

These are my measured timings on random input:\\
\begin{tabular}{lllll}
Method & N = $2^{20}$ & N = $2^{22}$ & N = $2^{24}$ & N = $2^{26}$ \\
    Det (g = 03) &  0.095 &  0.36 &  1.41 &  5.59 \\
    Det (g = 07) & 0.06 & 0.23 & 0.93 & 3.665 \\
    Det (g = 11) & 0.055 & 0.22 & 0.875 & 3.56 \\
    Det (g = 15) & 0.06 & 0.21 & 0.87 & 3.51 \\
    Det (g = 19) & 0.055 & 0.225 & 0.885 & 3.605 \\
    Det (g = 23) & 0.055 & 0.22 & 0.895 & 3.545 \\
             RND &  0.01 &  0.06 &  0.23 &  1.03 \\
         STL sel & 0.015 & 0.065 & 0.245 & 0.79 \\
        STL sort & 0.12 & 0.52 & 2.275 & 9.855 \\
       mergesort & 0.31 & 1.29 & 5.48 & 23.195
\end{tabular}

It seems clear, that the deterministic linear-time algorithm is not
the fastest method. It does however beat other sorting-based
algorithms. Of course the interesting observation is that the
deterministic approach is slower than the randomized.

Just to be sure, I measured the timings on pre-ordered input too. I
don't believe this will yield any significant change, since none of
the algorithms have longer expected running time here:\\
\begin{tabular}{lllll}
Method & N = $2^{20}$ & N = $2^{22}$ & N = $2^{24}$ & N = $2^{26}$ \\
    Det (g = 03) &  0.09 &  0.36 &  1.4 &  5.615 \\
    Det (g = 07) & 0.06 & 0.235 & 0.92 & 3.69 \\
    Det (g = 11) & 0.055 & 0.225 & 0.89 & 3.575 \\
    Det (g = 15) & 0.06 & 0.215 & 0.89 & 3.505 \\
    Det (g = 19) & 0.055 & 0.225 & 0.93 & 3.61 \\
    Det (g = 23) & 0.05 & 0.22 & 0.885 & 3.555 \\
             RND &  0.01 &  0.06 &  0.24 &  1.015 \\
         STL sel & 0.02 & 0.055 & 0.22 &  0.8 \\
        STL sort & 0.115 & 0.51 & 2.295 &   10 \\
       mergesort & 0.32 & 1.295 & 5.535 & 23.51
\end{tabular}

The overall result is the same. The deterministic method seems slower
than the randomized. However this could be due to an inefficient
implementation (they both have $O(n)$ theoretical running
time). However, I would recommend the randomized version - it's simple
to implement and seems very efficient. Of course, if the STL
implementation is available this would be the choice.

\section*{5. Random-Sampling Selection Parameters}
The random-sampling selection algorithm depends on the parameters
$\alpha$, $\beta$ and $\gamma$. It works by performing a three-way
partitioning on the input array, dividing it into 3 partitions. The
hope is that the sought element ends up in the middle section, while
the subsection defined by this section is easy. Hence, some problems
may arise:
\begin{enumerate}
\item The sought element is not in the middle section
\item The sought element is in the middle section, but this section is
  not small enough to solve efficiently
\end{enumerate}

Since the parameters $\alpha$, $\beta$ and $\gamma$ defines how the
input should be divided, these greatly affects the performance of the
algorithm.


\end{document}

%%% Local Variables:
%%% mode: latex
%%% TeX-master: t
%%% End:
