\documentclass[10pt,oneside,a4paper,final,english]{memoir}

\usepackage{lscape}
\usepackage{multicol}
%\usepackage{epic,eepic}
\usepackage{latexsym}
\usepackage{verbatim}
\usepackage{listings}
\usepackage{ulem}
\usepackage{hyperref}

\let\footruleskip\undefined
\usepackage{fancyhdr}
\usepackage[final]{fixme}

\let\fref\undefined
\usepackage[plain]{fancyref}

%% FOR LOOP
\usepackage{ifthen,calc}
\newcounter{myforloopcounter}
\newcommand{\forloop}[5][1]% 
{\setcounter{#2}{#3}% 
\ifthenelse{#4}% 
{#5%
  \addtocounter{#2}{#1}% 
  \forloop[#1]{#2}{\value{#2}}{#4}{#5}% 
}%
% Else 
{}%
}% 


%% USAGE
%\forloop[step]{counter}{initial_value}{conditional}{code_block} 
\usepackage[english]{babel}
\usepackage[utf8]{inputenc}

%\selectlanguage{danish}

\lstset{language=Python,basicstyle=\small,
  columns=fullflexible}


\usepackage[pdftex]{graphicx}

\DeclareGraphicsExtensions{.jpg .png .pdf}


\usepackage{amsmath}
\usepackage{latexsym}
\usepackage{amssymb}


\usepackage[osf,sc]{mathpazo}
\usepackage{microtype}
%\usepackage{fourier}
\linespread{1.05}

%\usepackage[charter]{mathdesign}
%\usepackage{lmodern}

%\usepackage{algorithmic}
%\usepackage{algorithm}

\usepackage{amsthm}


\theoremstyle{plain}  \newtheorem{definition}{Definition}
\theoremstyle{remark} \newtheorem{lemma}{Lemma}
\theoremstyle{plain}  \newtheorem{theorem}{Theorem}
\theoremstyle{remark}  \newtheorem{example}{Example}


\newcommand{\p}{\ensuremath{^\prime}}
\DeclareGraphicsExtensions{.jpg, .eps, .png}
%%% Local Variables:
%%% mode: plain-tex
%%% TeX-master: "../master"
%%% End:

%\usepackage{algorithmic}
\usepackage{algorithm}
\usepackage[sectionbib,square]{natbib}
%\bibpunct{(}{)}{,}{a}{}{}
\setcitestyle{alpha}
%\setcitestyle{numbers,aysep={},yysep={;}}

\usepackage{datetime}

\chapterstyle{thatcher}


\usepackage[autosize,outputdir=graphs/]{dot2texi}
\usepackage{tikz}
\usetikzlibrary{shapes,arrows}


%\pagestyle{fancy}
\begin{document}
  \fontencoding{T1}
%  \fontseries{m}
%  \fontshape{n}
%  \fontsize{12}{15}
%  \selectfont

\newcommand{\ct}{\texttt}
%%%%%%%%%%%%%%%%%%%%%%%%%%%%%%%%%%%%%%%%%%%%%%%%%%%%%%%%
%                    Forside
%%%%%%%%%%%%%%%%%%%%%%%%%%%%%%%%%%%%%%%%%%%%%%%%%%%%%%%%
\makeatletter % open mode for reading @ signed variables
\def\maketitle{%
 \null
 \thispagestyle{empty}%
 \vfill
 \begin{center}\leavevmode
   \normalfont
   \LARGE{\raggedleft \@title\par}%
   \hrulefill\par
   \large{\raggedleft \subtitle\par}%
   \vskip 2cm
   {\today\par}%
 \end{center}%
 \vfill
 \begin{flushleft}
   {\large \@author } \\
   {\footnotesize \suplementInfo }
 \end{flushleft}
 \clearpage % Terminates the page here. Everything else vil be placed
            % on next page.
}
\makeatother % closing mode for reading @ signed variables
%%%%%%%%%%%%%%%%%%%%%%%%%%%%%%%%%%%%%%%%%%%%%%%%%%%%%%%%
%               Data til forside
%%%%%%%%%%%%%%%%%%%%%%%%%%%%%%%%%%%%%%%%%%%%%%%%%%%%%%%%
\title{Selection Problem $\cdot$ Week III}

\def\subtitle{Data Structures: Theory and Practice}

\author{Johan Sejr Brinch Nielsen} \def\suplementInfo{

\kern 5pt \hrule width 11pc \kern 5pt

\begin{tabular}{ll}
Email: & zerrez@diku.dk  \\
Cpr.:  & 260886-2547
\end{tabular}

% putter 5pt spacing oven over og neden under stregen
\kern 5pt \hrule width 11pc \kern 5pt

Dept. of Computer Science,  \\
University of Copenhagen

}


\maketitle
\newpage


\section*{1) Exercise 20.3-4}
I am to describe the following recurrence describing the space usage
of a van Emde Boas tree:
\[S(K) = (1 + \sqrt{K})\cdot S(\sqrt{K}) + O(\sqrt{K})\]

The representations are:
\begin{itemize}
\item $S(\sqrt{K})$: The space usage of vEB-tree of size $\sqrt{K}$.
\item $(1+\sqrt{K})$: The number of sub-trees the current tree
  contains. This is 1 for the summary tree and the $\sqrt{K}$ is for
  the sub-trees.
\item $O(\sqrt{K})$: The space usage of the current tree-node
  containing $\min$, $\max$ and $u$ and possibly references to the
  summary and sub-trees. However, this should not be necessary.
\end{itemize}

Next, I am to show, that the recurrence results in a running time of
$O(n)$. I prove this by complete (strong) induction:

Assume that $S(K) = O(K)$ for all $K < N$. Then
\[ S(N) = (1 + \sqrt{N}) \cdot O(\sqrt{N}) + O(\sqrt{N}) \Rightarrow \]
\[ S(N) = O(\sqrt{N}) + \sqrt{N}\cdot O(\sqrt{N}) + O(\sqrt{N})
\Rightarrow \]
\[ S(N) = O(N) + O(\sqrt{N}) \Rightarrow \]
\[ S(N) = O(N) \]



\section*{2) Problem 20-1: van Emde Boas Trees}
\subsection{a: What happens when inserting an element that already
  exists?}
Short answer: Nothing. The routine will recursively search through the
tree, looking for the element. At some point, the base case of $u
= 2$ will occur (a 2-vEB tree is the base case). Here we simply set
the bit representing the number being inserted ignoring that the bit
is already set. This does not change the state of the tree in any way.

\subsection{b: What happens when deleting a non-existing element?}
Short answer: Nothing. Again the routine searches the tree for the
element $x$. However, since the element is not in the tree, the base
case will either be $\min = \max = \text{nil}$ (empty tree) or $\min =
\max \neq \text{nil}$ (1 element, not $x$). In the first case nothing
happens. In the latter, a non-existing element is ``deleted'' by
setting a zero-bit to zero: nothing happens.


\end{document}


%%% Local Variables:
%%% mode: latex
%%% TeX-master: t
%%% End:
